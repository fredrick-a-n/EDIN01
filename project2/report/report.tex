\documentclass[12pt]{article}

% Packages
\usepackage{graphicx}
\usepackage{amsmath}
\usepackage{amsfonts}
\usepackage{amssymb}
\usepackage{hyperref}
\usepackage{listings}
\newenvironment{Courier}{\fontfamily{pcr}\selectfont}{\par}
% Title page
\title{Project 2: Shift Register Sequences}
\author{Group 06: Fredrick Nilsson}
\date{\today}

\begin{document}


\maketitle

\tableofcontents

\newpage

\section*{Home Exercise 1}¨
\subsection*{1.\ \(p(x) = x^4 + x^2 + 1 \text{ over } \mathbb{F}_2\)}

    Since \(x^4+x^2+1={(x^2+x+1)}^2\) it is not irreducible, and therefore not primitive.

\subsection*{2.\ \(p(x) = x^3 + x + 1 \text{ over } \mathbb{F}_3\)}

    Since \(x^3+x+1=x^3+3x^2+4x+4=(x+2)(x^2+x+2)\), it clearly has factors and is therefore not irreducible. Since it is not irreducible, it is not primitive.

\subsection*{3.\ \(p(x) = x^2 + \alpha^5x + 1 \text{ over } \mathbb{F}_{2^4}\text{, where } \alpha^4 + \alpha + 1 = 0\)}

If there is an \textit{i} such that \(p(a^i) = 0 \), then \(p(x)\) has a root, and is therefore not primitive nor irreducible.
\\
If \textit{i} = 6, then \(p(\alpha^6) = \alpha^{12} + \alpha^{11} + 1 = (\alpha^3 + \alpha^2 + \alpha + 1) + (\alpha^3 + \alpha^2 + \alpha) + 1 = 
2\alpha^3 + 2\alpha^2 + 2\alpha + 2 = 0\)
As shown, \(p(x)\) is reducible and therefore not primitive.
\section*{Lab Exercise 1}
    \subsection*{1.\ \(p(x) = x^{23} + x^5 + 1 \text{ over } \mathbb{F}_2\)}

        \begin{Courier}
        \hspace*{10pt} > Primitive(\(x^{23} + x^5 + 1\)) \textbf{mod} 2
        \\
        \hspace*{10pt} > True
        \end{Courier}
        Therefore \(p(x)\) is primitive, and therefore irreducible.

    \subsection*{2.\ \(p(x) = x^{23} + x^6 + 1 \text{ over } \mathbb{F}_2\)}

        \begin{Courier}
        \hspace*{10pt} > Primitive(\(x^{23} + x^6 + 1\)) \textbf{mod} 2
        \\
        \hspace*{10pt} > False
        \\
        \hspace*{10pt} > Irreduc(\(x^{23} + x^6 + 1\)) \textbf{mod} 2
        \\
        \hspace*{10pt} > False
        \end{Courier}
        Therefore \(p(x)\) is neither a primitive nor irreducible.

    \subsection*{3.\ \(p(x) = x^{18} + x^3 + 1 \text{ over } \mathbb{F}_2\)}

        \begin{Courier}
        \hspace*{10pt} > Primitive(\(x^{18} + x^3 + 1\)) \textbf{mod} 2
        \\
        \hspace*{10pt} > False
        \\
        \hspace*{10pt} > Irreduc(\(x^{18} + x^3 + 1\)) \textbf{mod} 2
        \\
        \hspace*{10pt} > True
        \end{Courier}
        Therefore \(p(x)\) is not a primitive, but it is irreducible.

    \subsection*{4.\ \(p(x) = x^{8} + x^6 + 1 \text{ over } \mathbb{F}_7\)}

        \begin{Courier}
        \hspace*{10pt} > Primitive(\(x^{8} + x^6 + 1\)) \textbf{mod} 7
        \\
        \hspace*{10pt} > False
        \\
        \hspace*{10pt} > Irreduc(\(x^{8} + x^6 + 1\)) \textbf{mod} 7
        \\
        \hspace*{10pt} > False
        \end{Courier}
        Therefore \(p(x)\) is neither a primitive nor irreducible.

    \subsection*{5.\ \(p(x) = x^{6} + \alpha^5x + 1 \text{ over } \mathbb{F}_{2^4}\)}

        \begin{Courier}
        \hspace*{10pt} > Primitive(\(x^{23} + x^6 + 1\)) \textbf{mod} 2
        \\
        \hspace*{10pt} > True
        \end{Courier}
        Therefore \(p(x)\) is primitive.

\section*{Home Exercise 2}
\(|\mathbb{F}_{2^4}| = 16 \implies \alpha^{15}  \equiv  1 \), therefore the possible orders for a polynomial consisting of one \(\alpha\) are all possible factors of 15, that is 1, 3, 5 and 15.
\subsection*{1.\ \(\alpha\)}

\(ord(\alpha) = n \implies \alpha^n \equiv \alpha^{15}  \implies n=15\).
The order of \(\alpha\) is 15.
\subsection*{2.\ \(\alpha^2\)}
\(ord(\alpha) = n \implies {\alpha^2}^n \equiv \alpha^{15} \implies n=15\).
The order of \(\alpha\) is 15.
\subsection*{3.\ \(\alpha^3\)}
\(ord(\alpha) = n \implies {\alpha^3}^n \equiv \alpha^{15} \implies n=5\).
The order of \(\alpha\) is 5.
\subsection*{4.\ \(\alpha^5\)}
\(ord(\alpha) = n \implies {\alpha^5}^n \equiv \alpha^{15} \implies n=3\).
The order of \(\alpha\) is 3.
\section*{Lab Exercise 2}

    \begin{Courier}
    \hspace*{10pt} > G18 := GF(2, 18, \(\alpha^{18}\) + \(\alpha^3\) + 1)
    \\
    \hspace*{10pt} > G18 := \(\mathbb{F}_{2^{18}}\)

    \end{Courier}
    \subsection*{1.\ \(\alpha\)}

        \begin{Courier}
        \hspace*{10pt} > a := G18:-ConvertIn(\(\alpha\));
        \\
        \hspace*{10pt} > ...
        \\
        \hspace*{10pt} > G18:-order(a)
        \\
        \hspace*{10pt} > 189
        \end{Courier}

    \subsection*{2.\ \(\alpha^2\)}

        \begin{Courier}
        \hspace*{10pt} > a := G18:-ConvertIn(\(\alpha^2\));
        \\
        \hspace*{10pt} > ...
        \\
        \hspace*{10pt} > G18:-order(a)
        \\
        \hspace*{10pt} > 189
        \end{Courier}

    \subsection*{3.\ \(\alpha^3\)}

        \begin{Courier}
        \hspace*{10pt} > a := G18:-ConvertIn(\(\alpha^3\));
        \\
        \hspace*{10pt} > ...
        \\
        \hspace*{10pt} > G18:-order(a)
        \\
        \hspace*{10pt} > 63
        \end{Courier}

    \subsection*{4.\ \(\alpha+\alpha^3\)}

    \begin{Courier}
        \hspace*{10pt} > a := G18:-ConvertIn(\(\alpha + \alpha^3\));
        \\
        \hspace*{10pt} > ...
        \\
        \hspace*{10pt} > G18:-order(a)
        \\
        \hspace*{10pt} > 262143
        \end{Courier}


\section*{Home Exercise 3}
\section*{Lab Exercise 3}
\section*{Home Exercise 4}
\section*{Lab Exercise 4}
\section*{Home Exercise 5}
\section*{Lab Exercise 5}
asdaadsssd\cite{wiki:shiftregister}


\newpage
\bibliographystyle{unsrt}
\bibliography{references}

\end{document}
